\chapter{Single-Stage Amplifiers}

%%%
\section{General Considerations}
从本章开始将进入模拟电路的部分。本章使用的符号如表 \ref{tab:chapter4-symbols} 所示。
\begin{table}[!htb]
    \centering
    \caption{单级放大电路符号表}
    \label{tab:chapter4-symbols}
    \begin{NiceTabular}{cccc}
        \Xhline{1pt}
        \textbf{Symbol} & \textbf{Meaning} & \textbf{Unit} & \textbf{Polarity} \\ \hline
        $v_[in]$ & 输入电压(Input voltage) & $\unit{\volt}$ & $\diagdown$ \\
        $v_[out]$ & 输出电压(Output voltage) & $\unit{\volt}$ & $\diagdown$ \\
        $i_[in]$ & 输入电流(Input current) & $\unit{\ampere}$ & $\diagdown$ \\
        $i_[out]$ & 输出电流(Output current) & $\unit{\ampere}$ & $\diagdown$ \\
        $R_[in]$ & 输入电阻(Input resistance) & $\unit{\ohm}$ & + \\
        $Z_[in]$ & 输入阻抗(Input impedance) & $\unit{\ohm}$ & $\diagdown$ \\
        $R_[out]$ & 输出电阻(Output resistance) & $\unit{\ohm}$ & + \\
        $Z_[out]$ & 输出阻抗(Output impedance) & $\unit{\ohm}$ & $\diagdown$ \\
        $R_[load]/R_[L]$ & 负载电阻(Load resistance) & $\unit{\ohm}$ & + \\
        $Z_[load]/Z_[L]$ & 负载阻抗(Load impedance) & $\unit{\ohm}$ & $\diagdown$ \\
        $A_[v]$ & 电压增益(Voltage gain/Gain) & 1 & $\diagdown$ \\
        $G_[m]$ & 跨导(Transconductance) & $\unit{\siemens}$ & $\diagdown$ \\
        $g$ & 转移电导(Transfer conductance) & $\unit{\siemens}$ & $\diagdown$ \\
        $i_[x]$ & 测试电流(Test current) & $\unit{\ampere}$ & $\diagdown$ \\
        $v_[x]$ & 测试电压(Test voltage) & $\unit{\volt}$ & $\diagdown$ \\
        \Xhline{1pt}
    \end{NiceTabular}
\end{table}

%%%%
\subsection{Equivalent of a One-Port Network}
首先回顾一下一端口电路的等效。对于一个线性含源一端口电阻电路,可以进行戴维宁等效(Thevenin Equivalent)或诺顿等效(Norton Equivalent)的转换,如图 \ref{fig:thevenin-norton-equivalent} 所示。
\begin{figure}[htp!]
    \centering
    \begin{subfigure}[b]{\textwidth}
        \centering
        \includegraphics[]{thevenin.pdf}
        \caption{戴维宁等效}
    \end{subfigure}
    \begin{subfigure}[b]{\textwidth}
        \centering
        \includegraphics[]{norton.pdf}
        \caption{诺顿等效}
    \end{subfigure}
    \caption{戴维宁等效与诺顿等效}
    \label{fig:thevenin-norton-equivalent}
\end{figure}

在戴维宁等效中,$V_[OC]$为开路电压,$R_[o]$为将N内所有的独立源置零后两端钮间的等效电阻。在诺顿等效中,$I_[SC]$为短路电流,$R_[o]$同样为两端钮间的等效电阻。
它们具有关系$V_[OC] = I_[SC] \cdot R_[o]$。

%%%%
\subsection{Input and Output Impedances}
对于放大电路,我们通常会将其分为很多级,每一级都有输入和输出。为了简化,我们对其进行等效,如图 \ref{fig:input-output-impedance} 所示。
\begin{figure}[!htb]
    \centering
    \includegraphics[]{input-output-impedance.pdf}
    \caption{放大电路示意图}
    \label{fig:input-output-impedance}
\end{figure}
需要注意,所谓的\textbf{放大电路,都是基于小信号而言的}。

\textbf{输入阻抗(Input Impedance)}$\bm{Z_{\mathrm{in}}}$是从放大电路输入端看进去的等效电阻,其值为输入端电压$v_[in]$与输入端电流$i_[in]$的比值,
\begin{equation}
    Z_[in] = \frac{v_[in]}{i_[in]}
\end{equation}
通常我们希望输入阻抗$Z_[in]$的值尽量大,这样才能使得放大电路从信号源索取的电流尽量小,降低功耗,同时使放大电路得到的输入电压$v_[in]$更接近信号源的输出电压$v_[s]$,信号电压的损失更小。

放大电路从输出端看进去,内部是一个含源一端口电路,可以等效为一个电压源与电阻串联或一个电流源与电阻并联。这个电阻就是\textbf{输出阻抗(Output Impedance)}$\bm{Z_{\mathrm{out}}}$,其值为将输入信号源置零后,输出端电压$v_[out0]$与输出端电流$i_[out0]$的比值,
\begin{equation}
    Z_[out] = \frac{v_[out0]}{i_[out0]}
\end{equation}
也可以用添加测试电流$I_[X]$或测试电压$V_[X]$的方式来测量输出阻抗。\\
通常我们希望输出阻抗$Z_[out]$的值尽量小,这样才能使得放大电路的负载能分到更大的电压,电路的带负载能力更强。

需要注意,输入阻抗是从输入端看去的,在等效电路中位于输出的位置,而输出阻抗是从输出端看去的,在等效电路中位于输入的位置。这在初学时容易记忆错误。
同时,\textcolor{red}{输入阻抗与输出阻抗之间\textbf{没有对称关系}},它们是两个不同的概念。

%%%%
\subsection{Basic Ideas of Analyzing Single-stage Amplifiers}
在模拟集成电路设计中,各级放大电路一般都是基于 MOSFET 来实现的。而为了实现高的输入阻抗,一般又会采用\textbf{栅极作为输入端}。

结合 MOSFET 的小信号模型(图 \ref{fig:mosfet-small-signal-model}),我们通常使用图 \ref{fig:amplifier-equivalent-circuit} 所示的等效电路来进行分析计算。
\begin{figure}[h!tb]
    \centering
    \includegraphics[]{amplifier-equivalent-circuit.pdf}
    \caption{放大电路等效电路}
    \label{fig:amplifier-equivalent-circuit}
\end{figure}
其中$g$为压控电流源(VCCS)的转移电导\footnote{转移电导$g$与放大电路的跨导$G_[m]$是两个不同的概念,转移电导是针对一个受控源而言的,而跨导是整个放大电路输出电流与输入电压的小信号比值。只有在负载$Z_[load]$为开路时两者才相等。在JianJun Zhou的ppt中二者没有很好地区分,容易误解,在此指出。},$Z_[out]$为输出阻抗,$Z_[load]$为负载阻抗,$v_[in]$为输入电压,$v_[out]$为输出电压。

注意图 \ref{fig:amplifier-equivalent-circuit} 中的输入端的等效电源是一个受控源而不是独立源,但这并不是误用诺顿等效。这个受控源的电流是$g v_[in]$,转移电导$g$是由电路的静态工作点(大信号)决定的,而输入电压$v_[in]$是由上一级电路决定的,因此这个受控源的电流实际不受这一级小信号放大电路本身的影响,相对于当前的小信号电路是独立的,是一个“独立源”。

将单级放大电路简化为图 \ref{fig:amplifier-equivalent-circuit} 所示的形式后,容易看出其\textbf{增益(Gain)$A_[v]$}为
\begin{equation}
    A_[v] = \frac{v_[out]}{v_[in]} = -g (Z_[out] // Z_[load])
    \label{eq:amplifier-gain}
\end{equation}
式 \ref{eq:amplifier-gain} 非常重要,在后续的分析中会经常用到。有时我们在计算时会忽略负号,仅考虑增益的绝对值(电压放大的倍数)。

这里说明几个初学小信号电路图时令人困惑的问题(有些大信号电路图也有这些问题)。
\begin{itemize}
    \item \textbf{小信号输入和输出都是相对于地的}。小信号输入电压$v_[in]$和输出电压$v_[out]$都是相对于地的电压。很多时候在图中输入输出电压只接入了一个端钮,例如图 \ref{fig:amplifier-equivalent-circuit} 中的绿色端钮。我们应该知道另一个端钮是没有画出的接地点。

    \item \textbf{有些导线中是没有电流的,它只表示两个节点的电位相等}。很多时候,为了表示多个节点都接地,我们会用导线将这些节点相连避免在图中绘制多个接地点。这些导线中是没有电流的,它只表示两个节点的电位相等,在分析电流时可直接将其忽略。图 \ref{fig:amplifier-equivalent-circuit} 中的蓝色导线就是一个例子。

    \item \textbf{有些端钮是悬空的,只代表一个电压探针}。有些时候,为了简化图形,我们会将某些端钮悬空,引出一条悬空导线并给出一个电压值标签,代表一个电压探针。例如图 \ref{fig:amplifier-equivalent-circuit} 中的$v_[out]$。与探针相连的导线中是没有电流的,只是为了作图的方便引出一条导线。图 \ref{fig:amplifier-equivalent-circuit} 中的红色导线中是没有电流的。但需要注意,并不是所有这样的端钮都代表探针,例如一些大信号电路图中的$V_[DD]$就不是探针,它是一个电源,与其相连的导线中是有电流的。
\end{itemize}
这些特殊情况并不会在图中标出,需要我们自己去判断理解。

在很多分析(尤其是练习题)中,我们只考虑一级放大电路,不关心前后级的连接。这时我们就不考虑输出端的负载阻抗$Z_[load]$,认为其接在开路上,$v_[out]$是一个悬空的端钮。
这时我们对图 \ref{fig:amplifier-equivalent-circuit} 进行进一步化简,得到图 \ref{fig:amplifier-equivalent-circuit-simplified} 所示的等效电路。
\begin{figure}[h!tb]
    \centering
    \includegraphics[]{amplifier-equivalent-circuit-simplified.pdf}
    \caption{放大电路等效电路(负载为开路)}
    \label{fig:amplifier-equivalent-circuit-simplified}
\end{figure}
其中$G_[m]$为跨导。当输出端的负载为开路时,这级放大电路的跨导$G_[m]$就等于图 \ref{fig:amplifier-equivalent-circuit} 中的转移电导$g$。
此时增益$A_[v]$为
\begin{equation}
    A_[v] = \frac{v_[out]}{v_[in]} = -G_[m] Z_[out]
    \label{eq:amplifier-gain-simplified}
\end{equation}
从式 \ref{eq:amplifier-gain-simplified} 中可以看出,要求出放大电路的增益,只需要求出跨导$G_[m]$和输出阻抗$Z_[out]$即可,如图 \ref{fig:get-gm-zout} 所示\footnote{为与\cite{Analog-CMOS}保持一致,此图中$i_[out]$, $i_[x]$方向与图 \ref{fig:amplifier-equivalent-circuit-simplified} 中不同,方向向左。这两种方向定义均可,但需要保持求取$G_[m]$的$i_[out]$方向与求取$Z_[out]$的$i_[x]$的方向一致,避免出现负号的差异。}。
\begin{itemize}
    \item \textbf{求取}$\bm{G_{\mathrm{m}}}$\textbf{时,将小信号输出电压}$\bm{v_{\mathrm{out}}}$\textbf{置零,并假设输出电流}$\bm{i_{\mathrm{out}}}$\textbf{存在}。$G_[m] = \eval{\dfrac{\partial I_[out]}{\partial V_[in]}}_{V_[out]=\text{const}}$。
    \item \textbf{求取}$\bm{Z_{\mathrm{out}}}$\textbf{时,将小信号输入电压}$\bm{v_{\mathrm{in}}}$\textbf{置零,并假设在输出端接入测试电压源}$\bm{v_{\mathrm{x}}}$。$Z_[out] = \eval{\dfrac{\partial V_[out]}{\partial I_[out]}}_{V_[in]=\text{const}}$。
\end{itemize}
\begin{figure}[htp!]
    \centering
    \begin{subfigure}[b]{0.48\textwidth}
        \centering
        \includegraphics[]{cal_gm.pdf}
        \caption{求取$G_[m]$}
    \end{subfigure}
    \begin{subfigure}[b]{0.48\textwidth}
        \centering
        \includegraphics[]{cal_zout.pdf}
        \caption{求取$Z_[out]$}
    \end{subfigure}
    \caption{求取$G_[m]$和$Z_[out]$的等效电路图}
    \label{fig:get-gm-zout}
\end{figure}
请牢记这两点,在后续的分析中会继续强化这种方法。需要注意的是,在\textcolor{deep-blue}{求取$G_[m]$和$Z_[out]$时},我们都\textcolor{deep-blue}{假设} $v_[out]$处存在电流,这时的\textcolor{deep-blue}{负载不再为开路},可以理解为分别接入了一个测试电流源$I_[x]$和测试电压源$V_[x]$。

%%%%
\subsection{``Common'' What?}
在分析放大电路时,我们经常会遇到“共X放大电路”(Common Xxx)的概念,例如“共源放大电路”(Common Source Amplifier,CS Amplifier),“共栅放大电路”(Common Gate Amplifier,CG Amplifier),“共漏放大电路”(Common Drain Amplifier,CD Amplifier)等等。这里的“共”到底是什么意思呢?

首先说明,“共X放大电路”是对于单个 MOSFET 而言的。一个 MOSFET 有3个可用的 terminal,是栅极(Gate),漏极(Drain)和源极(Source)。在单级放大电路中,我们通常将其中一个 terminal 作为输入端,另一个 terminal 作为输出端,剩余的一个 terminal 作为公共端(Common)。

\textbf{公共端需要接地或接在一个直流电压上},在小信号分析中接地。我们之前提到小信号分析中输入电压和输出电压都是相对于地的,正好就是相对于这个公共端而言的,这也就是“共”的含义,表明输入和输出电压的值的公共参考点。

\begin{enumerate}
    \item 共源级。显然公共端为源极。剩余两个 terminal,为了更大的输入阻抗,取栅极为输入端,漏极为输出端。图 \ref{fig:common-source-amplifier} 为一个典型的共源级放大电路。
            \begin{figure}[htp!]
                \centering
                \includegraphics[]{common-source-amplifier.pdf}
                \caption{共源级放大电路}
                \label{fig:common-source-amplifier}
            \end{figure}
    \item 共漏级。显然公共端为漏极。剩余两个 terminal,为了更大的输入阻抗,取栅极为输入端,源极为输出端。图 \ref{fig:common-drain-amplifier} 为一个典型的共漏级放大电路。
            \begin{figure}[htp!]
                \centering
                \includegraphics[]{common-drain-amplifier.pdf}
                \caption{共漏级放大电路}
                \label{fig:common-drain-amplifier}
            \end{figure}
    \item 共栅级。显然公共端为栅极。剩余两个 terminal,在实际应用中两种接法都有。图 \ref{fig:common-gate-amplifier} 展示了两种接法的共栅级放大电路。
            \begin{figure}[htp!]
                \centering
                \begin{subfigure}[b]{0.48\textwidth}
                    \centering
                    \includegraphics[]{common-gate-amplifier-1.pdf}
                    \caption{共栅级放大电路(漏极输入,输入电阻大)}
                \end{subfigure}
                \begin{subfigure}[b]{0.48\textwidth}
                    \centering
                    \includegraphics[]{common-gate-amplifier-2.pdf}
                    \caption{共栅级放大电路(源极输入,输入电阻小)}
                \end{subfigure}
                \caption{共栅级放大电路}
                \label{fig:common-gate-amplifier}
            \end{figure}
\end{enumerate}

此外,对于存在多个 MOSFET 的放大电路,有时也会将其拆成2个进行命名,例如图 \ref{fig:CS-CG} 所示的“共源共栅级”(Common Source Common Gate Amplifier,CSCG Amplifier)。
\begin{figure}[htp!]
    \centering
    \includegraphics[]{CS-CG.pdf}
    \caption{共源共栅级放大电路}
    \label{fig:CS-CG}
\end{figure}
其中 M1 为共源级,而 M2 为共栅级。

%%%
\section{Common-Source Stage}

%%%
\section{Common-Drain Stage / Source Follower}

%%%
\section{Common-Gate Stage}

%%%
\section{Cascode Stage}
