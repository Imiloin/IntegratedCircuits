%!TEX TS-program = xelatex
%!TEX encoding = UTF-8

% 使用自定义的文档类 AJbook.cls. 自动载入 xeCJK. 需要额外档案如下:
% font-setup-open.tex, coverpage.tex, titles-setup.tex, mycommand.sty, myarrows.sty
% 文档类选项 (key/val 格式):
% draftmark = true (未定稿, 底部显示日期) 或 false (成品), 默认 false,
% colors = true (链接带颜色无框) 或 false (黑色无框), 默认 true,
% traditional = true (繁体中文) 或 false (简体中文), 默认 false,
% coverpage = 封面档档名, 默认为空 (此时不制作封面), 一般是 .tex 档, 若为 *.pdf 的形式则直接引入 PDF 页面.
% fontsetup = 字体设置档档名,
% titlesetup = 章节格式设置档名.

% 注意: 如果文中未使用 \cite 和 \index 命令, 则可能报错.

% 需动用 imakeidx + xindy (两份索引), biblatex + biber (书目). 
% 索引用土法进行中文排序: 如 \index{zhongwen@中文} 等.
\PassOptionsToPackage{quiet}{fontspec}	% 避免 CJK 字体警告
\documentclass[
	draftmark = true,   % 草稿模式下, 每页底部将打印相关版本信息.
%	traditional = true,
%	colors = false,
	coverpage = IC_cover.pdf,
    geometry = a4,    % 版面设置, 目前仅容许 a4, b5, 默认 b5, 其它字串则不作自动设置.
	fontsetup = font-setup-open.tex,
	titlesetup = titles-setup.tex
]{AJbook}

\usepackage{bm}  % 数学粗体
\usepackage{mathrsfs}
\usepackage{stmaryrd} \SetSymbolFont{stmry}{bold}{U}{stmry}{m}{n}	% 避免警告 (stmryd 不含粗体故)
\usepackage{array}
\usepackage{makecell}	% 便于制表
\usepackage{tikz-cd}  % 使用 TikZ 绘图
\usetikzlibrary{positioning, patterns, calc, matrix, shapes.arrows, shapes.symbols}
\usetikzlibrary{decorations.pathreplacing,calligraphy}
\usepackage{braids}
\usepackage{tqft}
\usepackage{ytableau}
\usepackage{multirow}
\usepackage{threeparttable}	% 表格注释
\usepackage[inkscapearea=page]{svg}	% 插入 svg 图形
\graphicspath{{imgs/}}	% 设定图片目录
\usepackage{pythonhighlight}
\usepackage{subfigure} % 并排图片宏包
\usepackage{float} % 图片位置,H选项


\usepackage{annotate-equations} % 为公式添加注释
\renewcommand{\eqnannotationfont}{\ttfamily\footnotesize}  % 注释文本字体,替换为等宽字体
\tikzset{annotate equations/text/.style={font=\eqnannotationfont,color=WildStrawberry}}  % 注释文本样式,红色

% 定义颜色
\usepackage{xcolor}
\definecolor{yellow}{HTML}{b58900}
\definecolor{orange}{HTML}{cb4b16}
\definecolor{red}{HTML}{dc322f}
\definecolor{magenta}{HTML}{d33682}
\definecolor{violet}{HTML}{6c71c4}
\definecolor{blue}{HTML}{268bd2}
\definecolor{cyan}{HTML}{2aa198}
\definecolor{green}{HTML}{859900}
\definecolor{base03}{HTML}{002b36}
\definecolor{base02}{HTML}{073642}
\definecolor{base01}{HTML}{586e75}
\definecolor{base00}{HTML}{657b83}
\definecolor{base0}{HTML}{839496}
\definecolor{base1}{HTML}{93a1a1}
\definecolor{base2}{HTML}{eee8d5}
\definecolor{base3}{HTML}{fdf6e3}
\definecolor{LemonChiffon}{HTML}{fffacd}  % 柠檬绸色
\definecolor{Teal}{HTML}{008080}  % 青色
\definecolor{Olive}{HTML}{808000}  % 橄榄色

% PGF plots 用于封面绘制
\usepackage{pgfplots}
\pgfplotsset{compat=newest}

% 可以修改章节编号的深度,给 subsubsection 编号
\setcounter{secnumdepth}{3}

% 必要时仅引入部分档案
% \includeonly{}

% 生成索引: 选用 xindy + imakeidx
\usepackage[xindy, splitindex]{imakeidx}
\makeindex[
	columns=2,
	program=truexindy,
	intoc=true,
	options=-M texindy -I xelatex -C utf8,
	title={Keyword Index}]	% 名词索引

\usepackage[unicode, bookmarksnumbered]{hyperref}	% 启动超链接和 PDF 文档信息所需
% 设置 PDF 文件信息
\hypersetup{
	pdfauthor = {Imiloin},
	pdftitle = {AJbook 文档类模板},
	pdfkeywords = {Template},
	CJKbookmarks = true}

% 增加表格高度
\renewcommand*\arraystretch{1.5}

% 自订公式的编号 (非必要)
\numberwithin{equation}{section}
\renewcommand{\theequation}{\thesection--\arabic{equation}}

% 自订 figure 的编号 (非必要)
%\numberwithin{figure}{section}
%\renewcommand{\thefigure}{\thechapter--\arabic{figure}}

% 自订 table 的编号 (非必要)
%\numberwithin{table}{section}
%\renewcommand{\thetable}{\thechapter--\arabic{table}}

% 用 bibLaTeX 生成参考文献
% 载入书目库: 文档类业已引入 biblatex + biber
\usepackage[backend=biber]{biblatex}
\addbibresource{references.bib}

%%% 自定义部分
\usepackage{siunitx} % 单位,\qty{数值}{单位}

\usepackage{caption} % Required for customizing captions
\captionsetup{skip=6pt} % Vertical whitespace between figures/tables and the caption (default is 10pt)
\captionsetup{font={bf}} % Define caption font style

\usepackage{booktabs} % 三线表
\setlength{\aboverulesep}{10pt}
\usepackage{multirow} % 表格中的多行合并
\usepackage{makecell} % 表格中的换行
\usepackage{colortbl} % 表格中的单元格颜色,\rowcolor{颜色}、\columncolor{颜色}、\cellcolor{颜色}
\usepackage{nicematrix} % 更好的表格

\usepackage{parskip}
\setlength{\parskip}{0.5em} % 段落间距
\setlength{\parindent}{2em} % 段首缩进

\definecolor{deep-blue}{RGB}{0, 0, 176}
\definecolor{deep-red}{RGB}{176, 0, 0}
%%%

\begin{document}
	\frontmatter	% 制作封面和目录.
	% 注意: 为了简化, 本模板不含封面. 请通过文档类的参数进行设置.
	
	\mainmatter		% 正文开始:逐章引入 TeX 代码

	\chapter*{Introduction}
	本文档为本科生微电子专业课的个人笔记。\LaTeX 模板来自李文威老师的\href{https://github.com/wenweili/AlJabr-1}{《代数学方法》开源项目},做了一些小的修改。封面设计参考了 \href{https://dribbble.com/shots/21647162-Bento-boxes}{Nicolas Solerieu 在 Dribble 上的设计}。

	本项目遵循 \href{https://creativecommons.org/licenses/by-nc-sa/4.0/}{CC BY-NC-SA 4.0} 协议。
	\vspace{1em}
	\begin{flushright}\begin{minipage}{0.2 \textwidth}
		\begin{tabular}{c}
			{By Imiloin} \\
			\href{https://github.com/Imiloin}{Github profile}\\
		\end{tabular}
	\end{minipage}\end{flushright}


	% % % % % % % % % %
	\part{Semicounductor Physics}

	\chapter{Semicounductor}

%%%%
\section{固体量子理论}
\subsection{固体的能带结构}

简并半导体和非简并半导体是两种不同类型的半导体,其关键区别在于费米能级的占据情况。
简并半导体是指在绝对零度时,导带或价带中的所有能级状态都被电子占据的半导体。这种情况下,费米能级是简并的。
非简并半导体是指在绝对零度时,导带和价带都不是完全占据的半导体。此时费米能级是非简并的。

	\chapter[Improving Deep Neural Networks]{Improving Deep Neural Networks\setcounter{footnote}{0}\footnote{Hyperparameter Tuning, Regularization and Optimization}}


	% % % % % % % % % %
	\part{Semicounductor Devices}
	\chapter[MOSFET]{Metal-Oxide-Semiconductor \\ Field-Effect Transistor}

本章使用的符号如表 \ref{tab:mosfet-symbols} 所示。

\begin{table}[!htb]
    \centering
    \caption{MOSFET 符号表}
    \label{tab:mosfet-symbols}
    \begin{NiceTabular}{c|c|c}
        \Xhline{1pt}
        \textbf{Symbol} & \textbf{Meaning} & \textbf{Unit} \\ \hline
        $V_{\rm{DD}}$ & Drain voltage & $\unit{\volt}$ \\
        $V_{\rm{SS}}$ & Source voltage & $\unit{\volt}$ \\
        $V_{\rm{GG}}$ & Gate voltage & $\unit{\volt}$ \\
        $V_{\rm{BB}}$ & Bulk voltage & $\unit{\volt}$ \\
        $V_{\rm{TH}}$ & Threshold voltage & $\unit{\volt}$ \\
        $V_{\rm{DS}}$ & Drain-source voltage & $\unit{\volt}$ \\
        $V_{\rm{GS}}$ & Gate-source voltage & $\unit{\volt}$ \\
        $V_{\rm{GB}}$ & Gate-bulk voltage & $\unit{\volt}$ \\
        $V_{\rm{SB}}$ & Source-bulk voltage & $\unit{\volt}$ \\
        $I_{\rm{D}}$ & Drain current & $\unit{\ampere}$ \\
        $I_{\rm{S}}$ & Source current & $\unit{\ampere}$ \\
        $I_{\rm{G}}$ & Gate current & $\unit{\ampere}$ \\
        $I_{\rm{B}}$ & Bulk current & $\unit{\ampere}$ \\
        $I_{\rm{DS}}$ & Drain-source current & $\unit{\ampere}$ \\
        $I_{\rm{GS}}$ & Gate-source current & $\unit{\ampere}$ \\
        $I_{\rm{GB}}$ & Gate-bulk current & $\unit{\ampere}$ \\
        $I_{\rm{SB}}$ & Source-bulk current & $\unit{\ampere}$ \\
        $W$ & Channel width & $\unit{\meter}$ \\
        $L$ & Channel length & $\unit{\meter}$ \\
        $C_{\rm{OX}}$ & Oxide capacitance & $\unit{\farad}$ \\
        \Xhline{1pt}
    \end{NiceTabular}
\end{table}

\section[MOS]{Metal-Oxide-Semicounductor Capacitor}

\section{Structure of MOSFET}

MOSFET\footnote{本章中的MOSFET均为绝缘栅型,不考虑结型} 分为两类: \textbf{n-channel(n沟道)} 和 \textbf{p-channel(p沟道)}。其导电载流子分别为电子和空穴。
每一类又分为两种: \textbf{depletion(耗尽型)} 和 \textbf{enhancement(增强型)}。耗尽型的 MOSFET 的沟道是一直存在的,而增强型的 MOSFET 的沟道是需要外加电压才能形成的。

表 \ref{tab:mosfet-types} 总结了这四种 MOSFET 的特点。

\begin{table}[!htb]
    \centering
    \caption{MOSFET 的类型}
    \label{tab:mosfet-types}
    \begin{NiceTabular}{c|cccc}
        \Xhline{1pt}
        & \makecell[c]{{\textbf{N-channel}} \\ \textbf{Depletion}} & \makecell[c]{{\textbf{N-channel}} \\ \textbf{Enhancement}} & \makecell[c]{{\textbf{P-channel}} \\ \textbf{Depletion}} & \makecell[c]{{\textbf{N-channel}} \\ \textbf{Enhancement}} \\ 
        \hline
        \multirowcell{-3}{\textbf{Symbol}} & \includegraphics*[width=0.08\textwidth]{n_dep.pdf} & \includegraphics*[width=0.08\textwidth]{n_enh.pdf} & \includegraphics*[width=0.08\textwidth]{p_dep.pdf} & \includegraphics*[width=0.08\textwidth]{p_enh.pdf} \\
        \hline
        \textbf{Oxide} & positive ion doping & no ion doping & negative ion doping & no ion doping \\
        \textbf{Operation} & always on & off when $V_{\rm{GS}} = 0$ & always on & off when $V_{\rm{GS}} = 0$ \\
        \textbf{Source} & n+, electron out & n+, electron out & p+, hole out & p+, hole out \\
        \textbf{Drain} & n+, electron in & n+, electron in & p+, hole in & p+, hole in \\
        \textbf{Gate} & usually VDD & usually VDD & usually GND & usually GND \\
        \textbf{Body} & p, usually GND & p, usually GND & n, usually VDD & n, usually VDD \\
        $\bm{V_{\rm{TH}}}$ & negative & positive & positive & negative \\
        $\bm{I_{\rm{D}}}$ & positive & positive & negative & negative \\
        $\bm{V_{\rm{DS}}}$ & positive & positive & negative & negative \\
        $\bm{V_{\rm{GS}}}$ & > $V_{\rm{TH}}$ & > $V_{\rm{TH}}$ & < $V_{\rm{TH}}$ & < $V_{\rm{TH}}$ \\
        \multirowcell{-3}{\textbf{Output} \\ \textbf{Characteristics}} & \includegraphics*[width=0.17\textwidth]{n_dep_out.pdf} & \includegraphics*[width=0.17\textwidth]{n_enh_out.pdf} & \includegraphics*[width=0.17\textwidth]{p_dep_out.pdf} & \includegraphics*[width=0.17\textwidth]{p_enh_out.pdf} \\
        \multirowcell{-3}{\textbf{Transfer} \\ \textbf{Characteristics}} & \includegraphics*[width=0.17\textwidth]{n_dep_tran.pdf} & \includegraphics*[width=0.17\textwidth]{n_enh_tran.pdf} & \includegraphics*[width=0.17\textwidth]{p_dep_tran.pdf} & \includegraphics*[width=0.17\textwidth]{p_enh_tran.pdf} \\
        \Xhline{1pt}
    \end{NiceTabular}
\end{table}

在集成电路设计中,我们通常使用\textbf{增强型 MOSFET},因为它的沟道是需要外加电压才能形成的,可以更好地控制其电流。而耗尽型 MOSFET 的沟道是一直存在的,所以它的电流无法被控制。在不作特殊说明的情况下,本章中的 MOSFET 均为增强型。



	% % % % % % % % % %
	\part{Analog Integrated Circuits}

	% % % % % % % % % %
	\part{Digital Integrated Circuits}

	\appendix
\chapter{Appx Title}
Nothing here yet.


	% % % % % % % % % %
	\backmatter
	% 使用 bibLaTeX 制作书目
	\nocite{*}	% 列出所有参考文献, 即使未在正文中引用
	\printbibliography[heading=bibintoc]
	
	% 图, 表索引. 可有可无.
	\listoffigures
	\listoftables

	% 制作索引 (用 imakeidx 的功能可以制作多份)
	{\footnotesize
	\indexprologue{中文术语按汉语拼音排序.}
	\printindex}

\end{document}
